\documentclass[11pt]{scrartcl}

\usepackage[utf8]{inputenc} % Input encoding
\usepackage[T1]{fontenc} % Font encoding
\usepackage{siunitx} % Provides the \SI{}{} and \si{} command for typesetting SI units
\usepackage{graphicx} % Required for the inclusion of images
\usepackage{natbib} % Required to change bibliography style to APA
\usepackage{amsmath} % Required for some math elements 
\usepackage{lmodern} % Font used in the document
\usepackage{hyperref} % To add link in the document
\usepackage[headsepline]{scrpage2} % For headers and footers with KOMA classes
\usepackage{todonotes}
\usepackage{enumitem} % Enumeration package
\usepackage[english]{babel} 

\synctex=1 % For syncing skim with Emacs

\setlength\parindent{0pt} % Removes all indentation from paragraphs

\renewcommand{\labelitemi}{\textbullet} % make items in itemize environment bullets rather that dashes
\renewcommand{\labelenumi}{\alph{enumi}.} % Make numbering in the enumerate environment by letter rather than
                                % number (e.g. section 6)

%\usepackage{times} % Uncomment to use the Times New Roman font


%----------------------------------------------------------------------------------------
%	DOCUMENT INFORMATION
%----------------------------------------------------------------------------------------

\addtokomafont{disposition}{\normalfont\bfseries} % Make title/sections/subsections font the same as the rest
                                % of the document

\title{Philip Morris: Data Analysis Improvement of Ciliary Beating of 3D Epithelial Tissue\\\vspace{1cm}Final report\vspace{1cm}} % Title

\author{Simon Jenni \& Laurent Hayez} % Author name

\date{\today} % Date for the report

\begin{document}

%----------------------------------------------------------------------------------------
%	HEADERS AND FOOTERS
%----------------------------------------------------------------------------------------

\setfootsepline[text]{.4pt}

\pagestyle{scrheadings}
\automark[section]{section}
\ihead{{\sc Data Analysis Improvement of Ciliary Beating of 3D Epithelial Tissue}}
\ohead{}
\chead{}
\ifoot[Final report]{{\sc Final report}}
\ofoot[\pagemark]{\pagemark}
\cfoot[]{}

\maketitle % Insert the title, author and date

\thispagestyle{empty}

\begin{center}
\begin{tabular}{l r !{\textendash} l}
Contact: & Simon Jenni & \href{mailto:simujenni@students.unibe.ch}{simujenni@students.unibe.ch} \\ % Partner names
& Laurent Hayez: & \href{mailto:laurent.hayez@unine.ch}{laurent.hayez@unine.ch}\\
Supervisor: & Patrice Leroy: & \href{mailto:Patrice.Leroy@pmi.com)}{Patrice.Leroy@pmi.com} % Instructor/supervisor
\end{tabular}
\end{center}

% If you wish to include an abstract, uncomment the lines below
% \begin{abstract}
% Abstract text
% \end{abstract}

\vspace{1cm}
\renewcommand{\contentsname}{Table of contents}
\tableofcontents



%----------------------------------------------------------------------------------------
%	SECTION 1
%----------------------------------------------------------------------------------------

\section{Project Description}

\subsection{Project Context}

Philip Morris International (PMI) is a global cigarette and tobacco company with headquarters in Lausanne. The
research and development program of PMI focuses on the development of products with the potential to reduce
the risk of tobacco related diseases. To this end, new products are tested against ordinary cigarettes by
exposing human tissue cultures to smoke or aerosol of both products. The effect of the exposure is then
analysed by observing different features of the tissue, one of which is the ciliary beating.


\subsection{Goals and Objectives of the Project}

The goal of this project is to implement a tool for the automatic analysis of ciliary beating in tissue
movies. Concretely the objectives are the following:
\begin{itemize}
\item Allowing batch processing of video-data contained in a folder (including subfolders). 
\item Pre-processing of the video data in order to remove noise by smoothing with a customisable 3D kernel.
\item Scoring the tissue surface activity using simple descriptive statistics and storing the results in an
  activity image.
\item Determining the frequency distribution given per region of interest (ROI) and extracting the
  dominant frequency.
\item Processing should be possible on multiple scales, i.e. ROI of variable size. 
\item Illustrating the phase of the beating frequency.
\end{itemize}

Part of the project is also an evaluation of the performance of different techniques applied to the
problem and other research objectives such as:

\begin{itemize}
\item Evaluating the effect of the ROI size on performance.
\item Evaluating the probable shape of the beating pattern on a by beating movie basis.
\item Comparing different techniques for the frequency analysis (e.g. FFT and wavelet transform).
\end{itemize}

Given the scope of this project and the relatively large amount of objectives, it should be noted that some of
the objectives have been given a higher priority than others. The ultimate goal of this project is to provide
a tool for frequency analysis and task-priorities were therefore weighted with this goal in mind. This
means for example that de-noising, a whole subject on it's own, has not be studied and evaluated as
extensively as techniques for frequency analysis.

 
%----------------------------------------------------------------------------------------
%	SECTION 2
%----------------------------------------------------------------------------------------

\section{Methodology}

The implementation of the tool was carried out in Matlab. The decision to use Matlab has been taken in
agreement with the client and is based on the ease of handling image and video processing and the relatively
fast development time that Matlab provides.  Git has been used as version control tool.

Development has been done in an incremental and iterative fashion roughly following the SCRUM framework. The
objectives have been distributed across sprints of two weeks each. To keep track
of the progress and help manage the project we have been using Taiga, an open source project management
platform similar to JIRA. Both the client and project stakeholders were given access to Taiga and have been able
to follow the progress.

Testing of our implementation using synthetic test-data has been an integral part of
the development process to ensure correctness of the implementation. In order to maintain a high code-quality, code reviews by the other respective
developer have been performed for every task.

\subsection{State of the art}

The arguably most accurate method for analysing the ciliary beating frequency is the direct measurement from
high-speed video recordings. This is of course very time-consuming and therefore several automated methods
have been proposed. The most commonly used approaches for the automated analysis of ciliary beating are based
on using the Fast Fourier Transform (FFT) to analyse intensity-signals in a region of interest and has been the
principal approach and starting point for further exploration used in this project. Other methods such as
photomultiplier and modified photodiode techniques rely on different hardware and inputs and are therefore not
considered.

%----------------------------------------------------------------------------------------
%	SECTION 3
%----------------------------------------------------------------------------------------

\section{Realisation} 


%----------------------------------------------------------------------------------------
%	SECTION 4
%----------------------------------------------------------------------------------------

\section{Results}

Analysis of the tool, such as
\begin{itemize}
\item effect of ROI size on performance;
\item time consumption of the different techniques;
\item comparison of performance with different parameters (denoising, ...).
\end{itemize}


%----------------------------------------------------------------------------------------
%	SECTION 6
%----------------------------------------------------------------------------------------

\section{Recommendations}


\subsection{Statement of Recommendations}


\subsection{Limitations}


\subsection{Outstanding Issues and Perspective for Future Work}



%----------------------------------------------------------------------------------------
%	SECTION 7
%----------------------------------------------------------------------------------------

\section{[Other relevant section]}








%----------------------------------------------------------------------------------------


\end{document}